\documentclass{beamer}
\usepackage[latin1]{inputenc}
\usepackage{xcolor}
\usepackage{hyperref}
\usepackage{graphicx}
\usepackage{bbding} % For \HandRight
\usepackage{fancyvrb} % For \UseVerb \SaveVerb

\usetheme{Madrid}
\usecolortheme{default}

% Command that embeds a hand pointing to the right in a href label
\newcommand{\hrefhand}[2]{\raisebox{-0.4ex}{\HandRight}\,\href{#1}{#2}}

\title{COMP3320 Introduction to OpenGL}
\author{Alex Biddulph}
\institute{
    The University of Newcastle, Australia
    \and
    Based on the work provided at \url{www.learnopengl.com}
}
\date{Semester 2, 2021}

\begin{document}

\begin{frame}
    \titlepage
\end{frame}

\begin{frame}{What is OpenGL?}
    \begin{itemize}
        \item A standard, maintained by the \hrefhand{www.khronos.org}{\color{blue}Khronos Group},
              specifying how graphics operations should behave
              \begin{itemize}
                  \item Each operation is specified to generate a certain result
                  \item Graphics card manufacturers are free to implement operations however they please, provided the result
                        complies with the standard
              \end{itemize}
        \item An API (Application Programming Interface)
              \begin{itemize}
                  \item Operating system agnostic
                  \item Window system agnostic
              \end{itemize}
    \end{itemize}
\end{frame}

\begin{frame}{What is OpenGL?}
    \begin{itemize}
        \item A rendering library
              \begin{itemize}
                  \item An external library is needed to create a window that OpenGL can render on to
              \end{itemize}
        \item A state machine
              \begin{itemize}
                  \item OpenGL only knows about triangles
                  \item The current state tells OpenGL how to render those triangles
              \end{itemize}
    \end{itemize}
\end{frame}

\begin{frame}[fragile]{OpenGL Extensions}
    \begin{itemize}
        \item Graphics card manufacturers can implement extensions to the OpenGL specification
              \begin{itemize}
                  \item Not available on all devices
                  \item Need to query the drivers to see if a specific extension is available
                  \item Create a shader program using {\color{blue}\verb"glCreateProgram"}
                  \item Use {\color{blue}\verb"GL_ARB_extension_name"} to check for {\color{blue}\verb"extension_name"}
                  \item For example, {\color{blue}\verb"GL_ARB_transpose_matrix"} adds new functions allowing application matrices to be stored in row-major order
              \end{itemize}
    \end{itemize}
\end{frame}

\begin{frame}{Common OpenGL Libraries}
    \begin{itemize}
        \setlength{\itemindent}{1cm}
        \item[GLFW~\footnote{GLFW: \url{www.glfw.org}}:] Allows you to:
            \begin{itemize}
                \item Create and manage windows and OpenGL contexts
                \item Handle keyboard, mouse, and joystick inputs
            \end{itemize}
        \item[GLAD~\footnote{GLAD: \url{glad.dav1d.de}}:] OS-specific library abstracting away from the graphics card's
            implementation of the OpenGL functions
        \item[GLM~\footnote{OpenGL Mathematics: \url{glm.g-truc.net/0.9.9/index.html}}:] OpenGL C++ Mathematics library
            based on the OpenGL Shading Language (GLSL)
        \item[SOIL~\footnote{Simple OpenGL Image Library: \url{www.lonesock.net/soil.html}}:] Simple OpenGL Image
            Library - a small C library useful for uploading image textures into OpenGL
        \item[ASSIMP~\footnote{The Open-Asset-Importer-Lib: \url{www.assimp.org}}:] Open Asset Import Library - useful for loading 3D models from various common formats
    \end{itemize}
\end{frame}

\begin{frame}[fragile]{OpenGL Workflow with GLFW and GLAD}
    \begin{enumerate}
        \item Initialise GLFW and set OpenGL context version and profile to use
              \begin{itemize}
                  \item We will use OpenGL context version 3.3 and the core profile in these examples
              \end{itemize}
        \item Create a window and set its width, height, and title
        \item Make the window's context the main context for the current thread
        \item Initialise GLAD and set it up to find all of the OpenGL function pointers (this is OS specific)
        \item Set up callback functions to handle window resizing and user inputs
        \item Set up rendering objects and textures
        \item Enter a rendering loop that handles updating the screen
        \item Clean up
    \end{enumerate}
\end{frame}

\end{document}
