\documentclass{beamer}
\usepackage[latin1]{inputenc}
\usepackage{xcolor}
\usepackage{hyperref}
\usepackage{minted}
\usepackage{tikz}
\usepackage{graphicx}
\usepackage{bbding} % For \HandRight
\usepackage{fancyvrb} % For \UseVerb \SaveVerb

\usetheme{Madrid}
\usecolortheme{default}

% Command that embeds a hand pointing to the right in a href label
\newcommand{\hrefhand}[2]{\raisebox{-0.4ex}{\HandRight}\,\href{#1}{#2}}

\title{COMP3320 Introduction to OpenGL}
\author{Alex Biddulph}
\institute{
    The University of Newcastle, Australia
    \and
    Based on the work provided at \url{www.learnopengl.com}
}
\date{Semester 2, 2021}

\begin{document}

\begin{frame}
    \titlepage
\end{frame}

\begin{frame}[fragile]{More on Shaders}
    Typical structure of a GLSL shader
    \begin{minted}{glsl}
        #version version_number
        in type in_variable_name;

        out type out_variable_name;

        uniform type uniform_name;

        void main() {
          // process input(s) and do
          // some weird graphics stuff
          type weird_results = weird_processing;

          // output processed stuff to output variable
          out_variable_name = weird_results;
        }
    \end{minted}
\end{frame}

\begin{frame}[fragile]{More on Shaders}
    \begin{itemize}
        \item The first line specifies OpenGL version. In these examples we use version 3.3 using the core profile
              \mint{glsl}|#version 330 core|
        \item We can then specify inputs, outputs, and uniforms in any order.
        \item {\color{blue}\verb"type"} can be any of the following with {\color{red}\verb"n"} and
                  {\color{red}\verb"m"} one of $2, 3, \textrm{or }4$
              \begin{tabular}{c c c c c c}
                  {\color{blue}\verb"bool"}                                      & {\color{blue}\verb"int"}                                      & {\color{blue}\verb"uint"}                                      & {\color{blue}\verb"float"} & {\color{blue}\verb"double"} & {\color{blue}\verb"bvec"}{\color{red}\verb"n"} \\
                  {\color{blue}\verb"ivec"}{\color{red}\verb"n"} & {\color{blue}\verb"uvec"}{\color{red}\verb"n"} & {\color{blue}\verb"vec"}{\color{red}\verb"n"} &
                  {\color{blue}\verb"dvec"}{\color{red}\verb"n"} & {\color{blue}\verb"mat"}{\color{red}\verb"n"} &
                  {\color{blue}\verb"mat"}{\color{red}\verb"n"}{\color{blue}\verb"x"}{\color{red}\verb"m"}                                                                                                                                                                                                                                         \\
              \end{tabular}
        \item Arrays of these types are also permitted
        \item Vectors can be \emph{swizzled}
              \begin{minted}{glsl}
                vec3 pos;
                vec2 pos2 = pos.xy;
                vec4 pos4 = pos2.xyyx;
                vec3 pos3 = pos4.wzy;
            \end{minted}
        \item Finally we have the main function. This function must be in every shader
    \end{itemize}
\end{frame}

\begin{frame}[fragile]{Shader Inputs and Outputs}
    \SaveVerb{gl_Position}|gl_Position|
    \SaveVerb{glGetIntegerv}|glGetIntegerv|
    \begin{itemize}
        \item Inputs and outputs allow you to pass data in to and get data out of your shaders. They also facilitate
              passing data between shader stages
        \item In a vertex shader the inputs are also known as vertex attributes. OpenGL guarantees the existence of at
              least $16$ $4$-component vertex attributes.
              \begin{examples}
                  Query {\color{blue}\verb"GL_MAX_VERTEX_ATTRIBS"} using \hrefhand{https://www.khronos.org/registry/OpenGL-Refpages/gl4/html/glGet.xhtml}{\color{blue}\UseVerb{glGetIntegerv}} to find the maximum available on your hardware
              \end{examples}
        \item Built-in vertex shader output \hrefhand{https://www.khronos.org/registry/OpenGL-Refpages/gl4/html/gl_Position.xhtml}{\color{blue}\UseVerb{gl_Position}}
        \item Fragments shaders require a {\color{blue}\verb"vec4"} colour output variable
    \end{itemize}
\end{frame}

\begin{frame}[fragile]{Shader Inputs and Outputs}
    \SaveVerb{layout}|layout|
    \begin{itemize}
        \item Variable declarations can be decorated with the \hrefhand{https://www.khronos.org/opengl/wiki/Layout_Qualifier_(GLSL)}{\color{blue}\UseVerb{layout}} specifier
              \mint{glsl}|layout(location = 0) in vec3 aPosition;|
        \item Location layout simplifies linking vertex attributes and fragment shader outputs in the main program
        \item If an input and an output share the same type and name in two consecutive shaders, OpenGL will link them
              together
              \begin{minted}{glsl}
                // In the vertex shader
                out vec4 ourColour;
                // In the fragment shader
                in vec4 ourColour;
            \end{minted}
    \end{itemize}
\end{frame}

\begin{frame}[fragile]{Uniforms}
    \SaveVerb{glGetUniformLocation}|glGetUniformLocation|
    \SaveVerb{glUniform}|glUniform|
    \begin{itemize}
        \item Allows for setting arbitrary data in a shader program
        \item Global to a shader program
        \item Can be accessed by any shader at any stage in the program
        \item Maintain their value until you change it
        \item If you declare a uniform and don't use it in your GLSL shader code it will be silently removed
    \end{itemize}
    \begin{examples}
        Find the location of a uniform in the shader program using \hrefhand{https://www.khronos.org/registry/OpenGL-Refpages/gl4/html/glGetUniformLocation.xhtml}{\color{blue}\UseVerb{glGetUniformLocation}}
    \end{examples}
    \begin{examples}
        Set the value of a uniform using \hrefhand{https://www.khronos.org/registry/OpenGL-Refpages/gl4/html/glUniform.xhtml}{\color{blue}\UseVerb{glUniform}}
    \end{examples}
\end{frame}

\end{document}
